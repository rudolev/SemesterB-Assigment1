\documentclass{article}
\usepackage{amsmath}

\begin{document}

\title{Asymptotic Analysis of Recurrence Relations}
\author{}
\date{}
\maketitle

\section*{Solution}

\subsection*{1. \( T(n) = 4T(n/2) + n \)}
Using the Master Theorem, we compare with the recurrence form:
\[
T(n) = aT(n/b) + f(n),
\]
where \( a = 4 \), \( b = 2 \), and \( f(n) = n \).

We check the condition \( f(n) = O(n^c) \) where \( c = \log_b a = \log_2 4 = 2 \).
Since \( f(n) = O(n^1) \) and \( n^1 = O(n^2) \), Case 1 of the Master Theorem applies:
\[
T(n) = \Theta(n^2).
\]

\subsection*{2. \( T(n) = T(3n/7) + 1 \)}
This recurrence can be solved using the iterative method:
\begin{align*}
T(n) &= T(3n/7) + 1 \\
&= T(9n/49) + 1 + 1 \\
&= \dots \\
&= T(3^k n / 7^k) + k.
\end{align*}
Stopping when \( n = 1 \), we find \( k = \log_{7/3} n \), leading to:
\[
T(n) = \Theta(\log n).
\]

\subsection*{3. \( T(n) = 6T(n/2) + n^4 \log n \)}
Here, \( a = 6 \), \( b = 2 \), and \( f(n) = n^4 \log n \).
\[
c = \log_2 6 \approx 2.585.
\]
Since \( f(n) = n^4 \log n = \Omega(n^{4 - \epsilon}) \) for small \( \epsilon \), we use the Master Theorem’s Case 3 with the regularity condition, leading to:
\[
T(n) = \Theta(n^4 \log n).
\]

\subsection*{4. \( T(n) = 8T(n/2) + n^4 + 3n^3 \log n \)}
Here, \( a = 8 \), \( b = 2 \), and \( f(n) = n^4 + 3n^3 \log n \).
\[
c = \log_2 8 = 3.
\]
Since \( f(n) = n^4 + 3n^3 \log n = \Omega(n^{4-\epsilon}) \) for small \( \epsilon \), the dominating term is \( n^4 \). By the Master Theorem, Case 3 applies:
\[
T(n) = \Theta(n^4).
\]

\subsection*{5. \( T(n) = T(\sqrt{n}) + n \)}
Using the recursion tree method:
\begin{align*}
T(n) &= T(n^{1/2}) + n \\
&= T(n^{1/4}) + n^{1/2} + n \\
&= \dots \\
&= T(1) + n + n^{1/2} + n^{1/4} + \dots.
\end{align*}
The sum approximates to \( O(n) \), so we conclude:
\[
T(n) = \Theta(n).
\]

\end{document}
