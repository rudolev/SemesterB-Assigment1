\documentclass{article}
\usepackage{amsmath}

\begin{document}

\title{Asymptotic Analysis - Solutions}
\author{}
\date{}
\maketitle

\section*{1. \( n > \log n \) for all \( n \in N \) }

We prove that:

\[
n > \log n, \quad \forall n \in N
\]

\textbf{Proof:} Define the function:

\[
f(n) = n - \log n.
\]

For \( n = 1 \), we have:

\[
1 - \log 1 = 1 - 0 = 1 > 0.
\]

For \( n \geq 2 \), the derivative is:

\[
f'(n) = 1 - \frac{1}{n} > 0.
\]

Since \( f(n) \) is increasing and positive for small \( n \), it follows that \( f(n) > 0 \) for all \( n \in N \), proving the statement. 

\section*{2. Reflexivity: \( f(n) \in \Theta(f(n)) \) }

By the definition of \(\Theta\)-notation, \( f(n) \in \Theta(f(n)) \) if there exist constants \( c_1, c_2 > 0 \) and \( n_0 \) such that:

\[
c_1 f(n) \leq f(n) \leq c_2 f(n) \quad \forall n \geq n_0.
\]

By choosing \( c_1 = 1 \) and \( c_2 = 1 \), the inequality holds in all cases, proving reflexivity.

\section*{3. Transitivity: If \( f(n) \in \Theta(g(n)) \) and \( g(n) \in \Theta(h(n)) \), then \( f(n) \in \Theta(h(n)) \)}

By the definition of $\Theta$:

\[
c_1 g(n) \leq f(n) \leq c_2 g(n),
\]

\[
d_1 h(n) \leq g(n) \leq d_2 h(n).
\]

Now we will multiply the equation by the relevant constant on each side:

\[
(c_1 d_1) h(n) \leq f(n) \leq (c_2 d_2) h(n).
\]

Since \( c_1 d_1 \) and \( c_2 d_2 \) are positive constants, \( f(n) \in \Theta(h(n)) \).

\section*{4. Transpose Symmetry: \( f(n) \in O(g(n)) \iff g(n) \in \Omega(f(n)) \)}

By the definition of Big-O:

\[
f(n) \leq c g(n) \quad \text{for a constant } c > 0 \text{ and sufficiently large } n.
\]

Rewriting:

\[
g(n) \geq \frac{1}{c} f(n),
\]

which is the definition of \( g(n) \in \Omega(f(n)) \), proving the transpose symmetry. 

\section*{5. If \( f(n) \in \Omega(g(n)) \) for increasing monotone functions, then \( \log f(n) \in \Omega(\log g(n)) \)}

Since \( f(n) \in \Omega(g(n)) \), there exist constants \( c > 0 \) and \( n_0 \) such that:

\[
f(n) \geq c g(n), \quad \forall n \geq n_0.
\]

Taking logarithms:

\[
\log f(n) \geq \log (c g(n)) = \log c + \log g(n).
\]

Since \( \log c \) is a constant, we have:

\[
\log f(n) \in \Omega(\log g(n)).
\]

\section*{6. \( \log^k(n) \in O(n^\epsilon) \) for constants \( k \geq 1 \) and \( 1 > \epsilon > 0 \) }

For all constants \( k \geq 1 \) and \( 1 > \epsilon > 0 \), we need to prove that:

\[
\log^k(n) \in O(n^\epsilon).
\]

This means there exist constants \( c > 0 \) and \( n_0 \) such that:

\[
\log^k(n) \leq c n^\epsilon, \quad \forall n \geq n_0.
\]

By definition, a function \( f(n) \) belongs to \( O(g(n)) \) if there exist positive constants \( c \) and \( n_0 \) such that:

\[
f(n) \leq c g(n), \quad \forall n \geq n_0.
\]

Thus, we need to show that for sufficiently large \( n \), we can bound \( \log^k(n) \) by \( c n^\epsilon \).

It is known that for sufficiently large n:

\[
\log(n) \leq n^{\epsilon / k}.
\]

Raising both sides to the power of \( k \):

\[
\log^k(n) \leq \left(n^{\epsilon / k} \right)^k = n^\epsilon.
\]

Thus, we obtain:

\[
\log^k(n) \leq n^\epsilon.
\]

Thus, by the definition of Big-O, we conclude:

\[
\log^k(n) \in O(n^\epsilon).
\]

\section*{7. \( n^k \in O((1+\epsilon)^n) \) for constants \( k \) and \( \epsilon > 0 \) (4 points)}

We need to show:

\[
n^k = O((1+\epsilon)^n).
\]

For large \( n \), exponential functions grow faster than polynomial functions. Taking limits:

\[
\lim_{n \to \infty} \frac{n^k}{(1+\epsilon)^n} = 0.
\]

Thus, \( n^k \in O((1+\epsilon)^n) \), proving the claim.

\end{document}
